\section{系统相关技术介绍}

\subsection{Django介绍}

Django是由Python语言编写的开源免费的web应用程序的框架,可以在很短的时间内完成从概念到启动的web应用的开发。使用Django可以只关注于应用的开发无须重造多余的轮子。Django具有以下特点:

1. 快速性:Django从设计之初的目的就是为了使得开发人员可以快速地从应用概念到部署启动的过程。

2. 完整性:Django包含了一系列的额外的工具来处理常见的web开发任务。例如用户认证、内容管理、站点地图和RSS订阅等,这些工具都是开箱即用。

3. 安全性:Django的安全性可以帮助开发人员避免常见的安全漏洞,例如SQL注入、跨站脚本、跨站请求伪造和点击劫持等。它的用户认证系统提供了一个安全的方式来管理用户账户和密码。

4. 可扩展性:一些复杂的站点使用了Django的功能来迅速并且灵活的处理繁重的需求。

5. 灵活性:很多公司、组织和政府使用Django构造了各式各样的应用,例如内容管理系统、社交网络和科学计算平台等。

\subsection{Bootstrap介绍}

Bootstrap是一个免费开源的CSS框架,针对响应式、移动优先的前端web开发。它包含HTML、CSS和基于JavaScript的设计模板用来排版、表单、按钮、导航和其它界面部分等。Bootstrap只关注于简化web页面的开发,它最初用于web项目的目的是使用它的颜色选择、尺寸、字体和布局等。因此,开发人员的主要任务就是找到自己满意的色彩样式,并将其添加到自己的web项目中,提供了HTML元素的基本样式定义;除此之外,还可以利用CSS的类来进一步定义不同组件的样式。

\subsection{PostgreSQL介绍}

PostgreSQL是一个免费开源的关系型数据库,强调以高扩展性和SQL合规性为特点。Postgres以ACID为特性的事务管理、自动更新视图、物化视图、触发器、外键和存储过程为特性。它的设计主要用来处理一系列从单台机器到数据仓库或用很多并发用户访问的web站点等的工作量。Postgres具有丰富的数据类型,除了最基本的数据类型以外,还包括结构化的数据类型、文档数据类型和几何类型,还可以自定义复杂的数据类型;并且还能处理并发场景并且有着优越的性能。新项目Postgres的目的是通过增加最少的功能来完全支持所需要的类型。这些功能包括类型定义和完整描述数据关系的能力。Postgres的数据库能够“理解”关系,并可以使用一定的规则以自然方式在相关的表中检索信息。

\subsection{MTCNN介绍}
人脸检测和人脸对齐在一些不受控制的条件下有着各种各样的挑战,例如人脸姿势、照明情况和遮挡情况,然而,随着深度学习技术的飞速发展,人们可以利用端到端(end-to-end)框架,使用深度卷积网络模型,将这些问题一次性解决。Multi-Task Convolutional Neural Network(MTCNN\cite{mtcnn})是一种深度层叠的多任务框架,可以利用它们之间的相关性提升模型的表现性能。MTCNN引入了一种层叠结构,由三个阶段的深度卷积神经网络构成,以由粗到细的方式对人脸进行检测和人脸关键点定位。在训练学习过程,提出了一种新型的在线硬抽样挖掘策略(online hard sample mining strategy),可以在无手工样本选择的条件下自动地提升表现性能。

\subsection{Facenet介绍}

Facenet\cite{facenet}是Google团队在2015年CVPR会议上提出的人脸识别网络模型,该网络模型可以直接学习人脸图像到欧氏空间之间的映射,其中具有相似的人脸有着相近的距离,不同的人脸有着较远的距离。当这个空间被训练得到之后,人脸识别的任务就可以轻松的在Facenet系统上进行实现。该模型摒弃了之前深度学习的主流方法---对网络模型中间的瓶颈层(bottleneck layer)进行训练,而是通过深度卷积网络对输入图像处理得到的embedding向量进行训练优化。改变了训练目标,也需要对损失函数进行调整。Facenet中使用的是一种基于粗略对齐的匹配、不匹配三元组的损失函数,称之为triplet loss,在训练集产生三元组的方法叫做在线的三元组挖掘法(online triplet mining method)。使用这种方法可以非常高效的对人脸图像进行表示,可以仅仅使用128维的人脸embedding向量在LFW测试数据集上实现了99.63\%的准确率。