\section{绪论}
\zihao{-4}
\setlength{\baselineskip}{20pt}
\fancyhf{}
\renewcommand{\headrulewidth}{0.5pt}
\fancyhead[C]{\zihao{-5}基于FaceNet的智慧工厂管理系统}
\fancyfoot[C]{\zihao{-5}\tnr\thepage}
\pagenumbering{arabic}

\subsection{研究背景}

随着信息技术的高速发展,越来越多的行业领域开始与这一学科开始交叉挂钩,并且结合出了非常多样的理论知识与应用场景。关于信息管理系统,不管各行各业,无论是高校学籍档案、医疗设备信息还是农业大棚温度测控的场景,都离不开简单高效信息管理系统应用。而且在一些特定的应用场景,不同的信息管理系统所面向的功能也不尽相同,例如有些系统需要面对超大批量用户访问的高并发场景,而有些系统则需要在信息的安全性方面做进一步加强。然而在一些中小型企业工厂的信息管理方式中,例如对工厂员工的信息、客户信息、供应商信息、产品信息以及原材料信息的管理,工厂管理人员大部分往往都是通过手工的方式对信息进行管理,即使人们会用到一些办公软件,例如Word、Excel等;工厂与客户以及供应商的合作方式还都是以电话、微信等联系方式进行沟通交流,但这种方式毫无疑问是离散的、不统一的,并且信息的传递往往都并不是实时的,所以针对中小型工厂企业,设计实现一个通用的、简单易管理的信息管理系统往往有着很大的需求。
