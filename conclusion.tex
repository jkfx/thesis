\section{总结与展望}
% \addcontentsline{toc}{section}{总结与展望}

本文实现了一个基于Facenet的智慧工厂管理系统的设计与实现,整个管理系统使用Django框架开发,将数据存储到Postgresql数据库,Facenet网络模型使用PyTorch框架在Kaggle社区中的云计算平台进行训练,最终在MFR2上的准确率达到95.27\%。该系统针对中小型工厂企业管理人员对工厂信息管理方式,解决了日常信息管理效率低与信息传递不一致等问题,其中包括了对员工信息的统一管理、客户信息管理、供应商信息管理、产品与原材料信息的统一管理等。为了方便管理人员使用,在后台页面中,首先向登录的管理人员展示当前工厂部门组织的财务状况,并且生成财务折线图与饼图来对当前财务状况进行可视化展示。在该系统中,可以通过该系统来对指定员工下达工作任务与支付薪资等;通过指定客户来下达订单并且指派订单的送达情况,还可生成订单详情票据来进行打印存根;在供应商与原材料的信息管理中,可以指定供应商与原材料进行采购。

在员工考勤方面,除了向工厂部门管理人员展示员工签到记录,还可使用人脸识别对员工进行签到。并且在新冠疫情的冲击下,人们出行不可缺少的就是佩戴口罩,所以本系统所使用的Facenet网络模型在CASIA-Webmaskedface模拟佩戴口罩人脸数据集上进行了微调,使得该系统在人脸识别过程中,即使员工佩戴口罩也能实现较高的识别率。并且在本地员工人脸照片数据库文件中只需要一张照片,为了将传统的分类问题的闭集问题转换为开集问题,新添加员工信息与员工照片之后不需要再次训练,因为该模型计算员工图像的embedding编码与本地人脸数据库存在的员工图像只需要进行阈值比较便可实现人脸识别功能。

随着深度学习领域的飞速发展,各式各样的网络架构层出不穷,人们也不断冲击着世界最先进的模型性能,使得一些网络架构在特定任务上所达到的性能不断赶超人类的水平。尽管如此,一些最先进的网络模型在文章中报告的性能,由于复现模型的研究人员缺少数据集或对超参数的调节不那么准确,使得一些有着世界最先进性能的网络模型难以复现,但是随着研究人员的不断尝试与新提出的训练技巧,都可以使得一些网络模型发挥他们最大的性能。

在人脸识别领域,从一开始的分模块划分的人脸识别系统,到端到端的人脸识别系统,都经历着各式各样的改进方案。本文所使用的Facenet网络模型,是将输入的人脸图像映射到一个embedding空间中,这个空间可以看做一个超球面,判定人脸相似度的方法就是计算两个embedding之间的欧式距离,并且决策边界是通过softmax来进行计算的。而ArcFace\cite{arcface}的提出,使用的是基于角度的、余弦边距的损失函数,可以进一步提升人脸识别系统的准确率。并且人脸识别模型的训练的过程中,对于错误标注的图像标签在很大的程度上影响着模型的性能,所以构造出一个更大的、具有可以涵盖生活中各方面条件的环境与错误标注数量小的数据集,可以进一步提升模型的性能。